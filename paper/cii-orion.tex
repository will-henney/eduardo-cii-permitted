\documentclass[useAMS, usenatbib]{mnras}
\pdfsuppresswarningpagegroup=1
%
\usepackage[spanish,es-minimal,english]{babel}
\usepackage[utf8]{inputenc}
\usepackage{graphicx}

\usepackage{xcolor}
\usepackage{hyperref}
\usepackage{siunitx}
\usepackage{newtxtext}
\usepackage[stix2,smallerops]{newtxmath}
\usepackage{booktabs}
\hypersetup{colorlinks=True, linkcolor=blue!50!black, citecolor=black,
  urlcolor=blue!50!black}
\usepackage{etoolbox}
\robustify\bfseries
\robustify\itshape

\usepackage[shortlabels]{enumitem}

\bibliographystyle{mnras}

\sisetup{
  % explicit "+" is useful for velocities
  retain-explicit-plus = true,
  % prefer 10^6 over 1 x 10^6
  retain-unity-mantissa = false,
  % Use x +/- e instead of x(e)  
  separate-uncertainty = true,
  % Make sure to pick up bold font when used in section heading for instance
  detect-weight = true,
}
\DeclareSIUnit\msun{\text{M\ensuremath{_\odot}}}
\DeclareSIUnit\lsun{\text{L\ensuremath{_\odot}}}

%%
%% Will macros
%%
% A better \ion command that works in more circumstances
\newcommand\ION[2]{#1\,\scalebox{0.9}[0.8]{\uppercase{#2}}}
\newcounter{ionstage}
\renewcommand{\ion}[2]{\setcounter{ionstage}{#2}% 
  \ensuremath{\mathrm{#1\,\scriptstyle\Roman{ionstage}}}}
\newcommand\hii{\ion{H}{2}}
\newcommand\nii{[\ion{N}{2}]}
\newcommand\oiii{[\ion{O}{3}]}
\newcommand\oii{[\ion{O}{2}]}
\newcommand\Wav[1]{\ensuremath{\lambda #1}}
% Chemical formulae
\newcommand*\chem[1]{\ensuremath{\mathrm{#1}}}

\newcommand\Fion{\ensuremath{F_{\text{ion}}}}
\newcommand\ionpar{\ensuremath{U_{\text{ion}}}}

\title[Permitted C II lines in the Orion Nebula]{
  \boldmath
  Spatial mapping of \ion{C}{2} emission lines
  in the Orion Nebula:\\
  Recombination versus fluorescence
}

\author[Henney et al.]{%
  William J. Henney,\(^1\)\thanks{
    w.henney@irya.unam.mx
  }
  % J. Garc{\'{\i}}a-Rojas,\(^{2,3}\)
  J. E. M\'endez-Delgado,\(^{2,3}\)
  % C. Esteban,\(^{2,3}\)
  % \newauthor 
  % A. Mesa-Delgado,\(^{4}\)
  % K. Z. Arellano-C\'ordova,\(^{2}\)
  % and 
  % M. Núñez-Díaz\(^{4}\)
  \\
  \(^1\)\foreignlanguage{spanish}{
    Instituto de Radioastronomía y
    Astrofísica, Universidad Nacional Autónoma de México, Apartado
    Postal 3-72, 58090 Morelia, Michaoacán, Mexico}
  \\
  \(^2\)\foreignlanguage{spanish}{
    Instituto de Astrof\'isica de Canarias (IAC), E-38205 La Laguna, Spain}
  \\
  \(^3\)\foreignlanguage{spanish}{
    Departamento de Astrof\'isica, Universidad de La Laguna, E-38206 La Laguna, Spain}
  % \\
  % \(^4\)\foreignlanguage{spanish}{
  %    Domicilio Particular, Tenerife, Spain}
}
% These dates will be filled out by the publisher
\date{Accepted XXX. Received YYY; in original form ZZZ}

% Enter the current year, for the copyright statements etc.
\pubyear{2020}

\begin{document} 
\label{firstpage}
\pagerange{\pageref{firstpage}--\pageref{lastpage}}
\maketitle

\begin{abstract}
  Lots of \ion{C}{2} lines. 
\end{abstract}


\begin{keywords}
  Atomic physics
  -- H II regions
  -- Radiative transfer
  -- techniques: imaging spectroscopy
\end{keywords}

\maketitle

\section{Introduction}
\label{sec:introduction}


% \begin{figure}
%   \centering
%   \includegraphics[width=\linewidth]{figs/oii-emissivity-vs-t}
%   \caption{Emissivities of different \chem{O^{++}} lines versus temperature.}
%   \label{fig:emissivities}
% \end{figure}

Calculations of recombination spectrum \citep{Pequignot:1991a, Davey:2000a}.
Calculation of fluorescent and recombination spectrum for a particular source (PN IC~418) by \citet{Escalante:2012a}.
Similar calculation for Orion Nebula, but for the \ion{N}{2} lines \citep{Escalante:2005a}. 

Rich spectrum of \ion{C}{2} observed in stellar wind of Wolf Rayet star in LMC \citep{Williams:2021s}. 

\section{MUSE observations of Orion}
\label{sec:muse-observ-orion}

% \begin{figure}
%   \centering
%   \includegraphics[width=\linewidth]{figs/adal-slit6-oii-v1-annotated}
%   \caption{High-resolution ISIS-WHT spectra}
%   \label{fig:adal-pink-spectra}
% \end{figure}

% \begin{figure*}
%   \centering
%   \includegraphics[width=\linewidth]{figs/oii-v1-extraction-workflow}
%   \caption{Workflow for analyzing MUSE \ion{O}{2} spectral maps.  }
%   \label{fig:muse-workflow}
% \end{figure*}



\section{Discussion}
\label{sec:discussion}

% Other \ion{H}{2} regions.

% Planetary nebulae with moderate ADF.
% For instance the Ring Nebula \citep{Garnett:2001a} shows \ion{O}{2}~V1 peaking just inside [\ion{O}{3}],
% which is what one would expect from fluorescence. 


% Planetary nebulae with high ADF.
% I don't think there is any evidence for fluorescence being important in this case.
% \citet{Fang:2013a} have the lowest temperatures from \ion{O}{2} 3d--4f,
% which is consistent with cold clumps.

\section{Conclusions}
\label{sec:conclusions}


\bibliography{cii-orion-refs}


% Don't change these lines
\bsp	% typesetting comment
\label{lastpage}

\end{document}


%%% Local Variables:
%%% mode: latex
%%% TeX-master: t
%%% End:
